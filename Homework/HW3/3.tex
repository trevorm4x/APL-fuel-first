\documentclass{article}
\usepackage[margin=0.5in]{geometry}
\usepackage{titlesec}
\usepackage{ifthen}
\usepackage{fancyhdr}
\usepackage{xcolor}

% -------- %
% SECTIONS %
% -------- %
\newcounter{problemnumber}\setcounter{problemnumber}{1}
\titlespacing\section{0pt}{10pt}{0pt}   % Spacing between Problems
\titlespacing\subsection{0pt}{5pt}{0pt} % Spacing between Parts
\newcommand{\problem}[1][-1]{
    \setcounter{partnumber}{1}
    \ifnum#1>0
        \setcounter{problemnumber}{#1}
    \fi
    \section*{Problem \arabic{problemnumber}}
    \stepcounter{problemnumber}
}

\newcounter{partnumber}\setcounter{partnumber}{1}
\newcommand{\ppart}[1][-1]{
    \ifnum#1>0
        \setcounter{partnumber}{#1}
    \fi
    \subsection*{Part \parttype{partnumber}}
    \stepcounter{partnumber}
}

\newenvironment{question}{
    \color{gray}\itshape
    \vspace{5pt}
    \begin{tabular}{|p{0.97\linewidth}}
}{
    \end{tabular}\\[5pt]
}



% ------------- %
% HEADER/FOOTER %
% ------------- %
\setlength\parindent{0pt}
\setlength\headheight{30pt}
\headsep=0.25in
\pagestyle{fancy}
\lhead{\ifthenelse{\thepage=1}
    {\textbf{Trevor Smith} \\ \textbf{\writeday}}
}
\chead{\ifthenelse{\thepage=1}
    {\textbf{\huge{HOMEWORK \hwnumber}}}
    {\textbf{\large{HOMEWORK \hwnumber}}}
}
\rhead{\ifthenelse{\thepage=1}
    {\textbf{{\course}} \\ \textbf{Professor {\prof}}}
}
\cfoot{\thepage}
\renewcommand\headrulewidth{0.4pt}
\renewcommand\footrulewidth{0.4pt}



% ---------- %
% PARAMETERS %
% ---------- %
% \PARTTYPE:
% \Alph   := "Part A, Part B,  ..."
% \alph   := "Part a, Part b,  ..."
% \arabic := "Part 1, Part 2,  ..."
% \Roman  := "Part I, Part II, ..."
\newcommand\parttype{\Roman}

% \COURSE:
\newcommand\course{PHYS 3600}

% \HWNUMBER
\newcommand\hwnumber{3}

% \SEMESTER
\newcommand\semester{Summer 1 2021}

% \PROF
\newcommand\prof{Altunkaynak}

% \WRITEDAY
% \today is date of compilation, replace if writing due date rather than write date
\newcommand\writeday{\today}



%  ------- %
% DOCUMENT %
% -------- %
\begin{document}
\problem
\begin{question}
	Compare and contrast plasma emission and flourescence.
\end{question}
Plasma emission is caused by heating a sample to high temperatures causing 
excitation, where flourescence is caused by more direct excitation and can be
produced using light. In both cases the light produced is energy being ejected
from an electron transtitioning back to a lower energy state. Plasma emission
produces very tight bands. We interact with both, with flourescent bulbs and
neon lighting both using the different methods.

\problem
\begin{question}
	Describe the ruby crystal structure and its elements. Which element is responsible for the fluorescence and what does it replace?
\end{question}
Ruby crystal is a form of doped corundum, which is $Al_2O_3$ in a 
hexagonal-rhombohedral repeating lattic1e where every oxygen atom is connected
to 6 other oxygen atoms. The doping moleule chromium replaces
about 1 in 100 aluminum molecules, and is responsible for the flourescence of
the ruby. This is caused by three unpaired 3d electrons of $Cr^{3+}$ compared to
$Al^{3+}$ upsetting the perfect stability of the corundum crystal structure.

\problem
\begin{question}
	Briefly explain population inversion in a lasing medium. What does the lifetime of an energy level
have to do with it?
\end{question}
Population inversion what happens when most atoms in a material are in a higher
energy state instead of the natural lower energy state. 
The ability to generate a population inversion on a sample is dependent on the
lifetime of the energy levels available within the material. Typically in a 
laser the atoms are excited to a short-lived higher energy state which decays
quickly into a meta-stable energy state. Molecules in the meta-stable energy
state are then able to become the new dominant population, as we can 
continually pump in excitation light to bring the molecules to the higher-energy
state.

\problem
\begin{question}
	Briefly describe the differences between spontaneous emission and stimulated emission.
\end{question}
Most briefly, spontaneous emission happens spontaneously and stimulated emission
is stimulated. Importantly, stimulated emission is stimulated by another photon
matching the energy level difference, and results in the production of a photon
of the exact same wavelength, direction, and phase, contrasted with spontaneous emission
which is fully random in phase and direction.

\problem
\begin{question}
	What percentage of visible light is passed through a 3 mm thick window made of fused silica (SiO2)? \\
What percentage of light is passed through a 0.1 mm thick clear Mylar (polyethylene) plastic sheet? \\
Which transmits more light? Assume absorption is zero. Compute and explain.
\end{question}
$$ n_{silica} = 1.4585 $$
$$ n_{polyethelene} = 1.500 $$
$$ R = \left(\frac{1-n}{1+n}\right)^2 $$
$$ R_{silica} = 0.03478 $$
$$ R_{polyethelene} = 0.04 $$
When $\alpha$ = 0,
$$ T = (1-R)^2 $$
$$ T_{silica} = 93.2 \% $$
$$ T_{polyethelene} = 92.2 \% $$
Silica transmits more light, as we can see by comparing T. This makes sense 
given the larger difference in n going from vaccuum to polyethelene than 
vaccuum to silica.


\problem
\begin{question}
	In designing a laser diode (LD), what elements are required to produce blue light?
\end{question}
Gallium nitride is required to produce blue light.

\problem
\begin{question}
	Why do we need a green laster for the lab?
\end{question}
If we want to stimulate flourescence, it's much harder to use white light since
it is made up of a scattering of frequencies, and we may not be able to produce
a high enough energy light for the specific absorption we're looking for. 
A green laser does have a high enough energy to excite R-line flourescence.

\problem
\begin{question}
	Can you replace the green laser with a red or blue laser to excite the R-line fluorescence?
\end{question}
One explanation for why a red laser couldn't be used is that rubies, being red,
do not absorb red light. Another explanation is that for flourescence to occur, 
we must hit the sample with higher energy than will be emitted. Since the
atom will excite to a short-lived state, then decay to a meta-stable state
from which it will produce a photon of wavelength in the red light range,
if the original photon exciting the atom was also red light that would violate
conservation of energy as the atom lost more energy than it gained, due to the 
extra meta-stable state. \\
The energy of a photon is given by $hc/\lambda$, where hc is planck's constant * the speed of light in eV*nm. The lower the wavelength, therefore, the higher
the energy of the photon. Blue light is on the lower-wavelength end of the spectrum, as it is close to ultra-violet. This means that a blue laser would have 
higher energy light than a green laser, and could be used for this experiment.

\problem
\begin{question}
	Depending on the relative positions of a lens and an object, the image produced by a convex lens
could be real, imaginary or no image. What type of image is produced by the following: \\
a. the object is further than the focal length; \\
b. the object is at the focal length; \\
c. the object is less than the focal distance.
\end{question}
a. Real image
b. No image
c. Imaginary image

\problem
\begin{question}
	Briefly describe how a spectrometer works.
\end{question}
Light is passed through a narrow slit, serving as a short-pass filter.
The light is then collimated and passed through a grating. This grating 
disperses the light at different angles depending on its wavelength, like a 
prism. These components are then focused again onto a detector.







\end{document}
