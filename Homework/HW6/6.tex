\documentclass{article}
\usepackage[margin=0.5in]{geometry}
\usepackage{titlesec}
\usepackage{ifthen}
\usepackage{fancyhdr}
\usepackage{enumitem}
\usepackage{xcolor}

% -------- %
% SECTIONS %
% -------- %
\newcounter{problemnumber}\setcounter{problemnumber}{1}
\titlespacing\section{0pt}{10pt}{0pt}   % Spacing between Problems
\titlespacing\subsection{0pt}{5pt}{0pt} % Spacing between Parts
\newcommand{\problem}[1][-1]{
    \setcounter{partnumber}{1}
    \ifnum#1>0
        \setcounter{problemnumber}{#1}
    \fi
    \section*{Problem \arabic{problemnumber}}
    \stepcounter{problemnumber}
}

\newcounter{partnumber}\setcounter{partnumber}{1}
\newcommand{\ppart}[1][-1]{
    \ifnum#1>0
        \setcounter{partnumber}{#1}
    \fi
    \subsection*{Part \parttype{partnumber}}
    \stepcounter{partnumber}
}

\newenvironment{question}{
    \color{gray}\itshape
    \vspace{5pt}
    \begin{tabular}{|p{0.97\linewidth}}
}{
    \end{tabular}\\[5pt]
}



% ------------- %
% HEADER/FOOTER %
% ------------- %
\setlength\parindent{0pt}
\setlength\headheight{30pt}
\headsep=0.25in
\pagestyle{fancy}
\lhead{\ifthenelse{\thepage=1}
    {\textbf{Trevor Smith} \\ \textbf{\writeday}}
}
\chead{\ifthenelse{\thepage=1}
    {\textbf{\huge{HOMEWORK \hwnumber}}}
    {\textbf{\large{HOMEWORK \hwnumber}}}
}
\rhead{\ifthenelse{\thepage=1}
    {\textbf{{\course}} \\ \textbf{Professor {\prof}}}
}
\cfoot{\thepage}
\renewcommand\headrulewidth{0.4pt}
\renewcommand\footrulewidth{0.4pt}



% ---------- %
% PARAMETERS %
% ---------- %
% \PARTTYPE:
% \Alph   := "Part A, Part B,  ..."
% \alph   := "Part a, Part b,  ..."
% \arabic := "Part 1, Part 2,  ..."
% \Roman  := "Part I, Part II, ..."
\newcommand\parttype{\Roman}

% \COURSE:
\newcommand\course{PHYS 3600}

% \HWNUMBER
\newcommand\hwnumber{6}

% \SEMESTER
\newcommand\semester{Summer 1 2021}

% \PROF
\newcommand\prof{Altunkaynak}

% \WRITEDAY
% \today is date of compilation, replace if writing due date rather than write date
\newcommand\writeday{\today}



%  ------- %
% DOCUMENT %
% -------- %
\begin{document}

\problem
\begin{question}
	Write down the potential difference across the capacitor.
\end{question}
$$ V_c = \frac{Q}{C} $$
$$ V_c(t) = \frac{q(t)}{C} $$

\problem
\begin{question}
	Write down the potential difference across the resistor.
\end{question}
$$ V = -IR $$
$$ V_R(t) = I(t)R $$

\problem
\begin{question}
	Write down the voltage required to overcome the back e.m.f.
	$V_L(t)$ in the inductance (L).
\end{question}
$$ V_L(t) = -L\frac{dI}{dt} $$

\problem
\begin{question}
	Show that 
	$$ L \frac{dI}{dt} + RI + \frac{q}{C} = 0 $$
\end{question}
When the switch is in the right position, we can just use Kirchoff's loop
rule for the single loop.
$$ \frac{q(t)}{c} - (-I(t)R) - (-L \frac{dI}{dH}) = 0 $$
$$ L\frac{dI}{dt} + I(t)R + \frac{q(t)}{C} = 0 $$
$$ L \frac{dI}{dt} + RI + \frac{q}{C} = 0 $$

\problem
\begin{question}
	Show that blank
\end{question}

$$ L \frac{dI}{dt} + RI + \frac{q}{C} = 0 $$
$$ \frac{dI}{dt} + \frac{R}{L}I + \frac{q}{LC} = 0 $$
$$ 2k = \frac{R}{L} $$
$$ \omega^2 = \frac{1}{LC} $$
$$ \frac{dI}{dt} + 2k I + \omega^2 q = 0 $$
$$ I = \frac{dq}{dt} $$
$$ \frac{d^2q}{dt^2} + 2k\frac{dq}{dt} + \omega^2 q = 0 $$

\problem
\begin{question}
	Using the trivial(?) solution $q = Qe^{\alpha t}$, solve for $\alpha$.
\end{question}

$$q = Qe^{\alpha t}$$
$$\frac{dq}{dt} = \alpha Qe^{\alpha t}$$
$$\frac{d^2q}{dt^2} = \alpha^2 Qe^{\alpha t}$$

$$ \alpha^2 Qe^{\alpha t} + 2k\alpha Qe^{\alpha t} + \omega^2 Qe^{\alpha t} = 0 $$
$$ Qe^{\alpha t}\left(\alpha^2  + 2k\alpha + \omega^2 \right) = 0 $$
Solve quadratic
$$ \alpha = \frac{-2k \pm \sqrt{4k^2-4\omega^2}}{2} $$

\problem
\begin{question}
	Derive the equation for R that determines the onset of critical damping.
\end{question}
Zooming in on that quadratic,
$ \sqrt{4k^2-4\omega^2} $ must be 0. 
$$ 4k^2-4\omega^2 = 0 $$
$$ k^2 = \omega^2  $$
$$ \frac{R^2}{L^2} = \frac{1}{LC}  $$
$$ \frac{R^2}{L} = \frac{1}{C}  $$
$$ R^2 = \sqrt{\frac{L}{C}}  $$

\problem
\begin{question}
	For L=0.50 H and C=1.80e-9F, what is the maximum value for resistance
	so that the circuit is underdamped.
\end{question}

$ \sqrt{4k^2-4\omega^2} $ must be $<$ 0. 
$$ k^2-\omega^2<0 $$
$$ R^2 < \sqrt{\frac{L}{C}}  $$
$$ R^2 < \sqrt{\frac{0.50}{1.8 \mu F}}  $$
R must be les than 527 $\Omega$

\problem
\begin{question}
	If the resistance in an underdamped LRC oscillator is decreased by 
	a factor of two, from $R=2R_0$ to $R=R_0$, how much does the oscillation
	amplitude change over a time $t=2L/R_0$?
\end{question}
General solution is 
$$ y = e^{\gamma t} \left[ C_1cos(\beta t) + C_2 sin(\beta t) \right] $$
Solution to characteristic equation
$$ r = \gamma \pm \beta i $$
$$\alpha = -k \pm \sqrt{k^2-\omega^2} $$
$$ \sqrt{k^2-\omega^2} = \beta i $$
$$ \gamma = -k $$
$$ y = e^{-k t} \left[ C_1cos(\beta t) + C_2 sin(\beta t) \right] $$
$$ y = e^{-\frac{R}{2L} t} $$
\problem
\begin{question}
	If the resistance in an underdamped LRC oscillator is decreased by 
	a factor of two, from $R=2R_0$ to $R=R_0$, how much does the oscillation
	amplitude change over a time $t=2L/R_0$?
\end{question}
General solution is 
$$ y = e^{\gamma t} \left[ C_1cos(\beta t) + C_2 sin(\beta t) \right] $$
Solution to characteristic equation
$$ r = \gamma \pm \beta i $$
$$\alpha = -k \pm \sqrt{k^2-\omega^2} $$
$$ \sqrt{k^2-\omega^2} = \beta i $$
$$ \gamma = -k $$
$$ y = e^{-k t} \left[ C_1cos(\beta t) + C_2 sin(\beta t) \right] $$
$$ y = e^{-\frac{R}{2L} t} $$

$$ R = 2R_0 \rightarrow y = e^{-\frac{2R_0}{2L} t} =
e^{-\frac{2R_0}{2L} \frac{2L}{R_0}} = e^{-2} A $$
$$ R = R_0 \rightarrow y = e^{-\frac{R_0}{2L} t} =
e^{-\frac{R_0}{2L} \frac{2L}{R_0}} = e^{-1} A $$
Oscillation amplitude decreases by a factor of 1/e

\problem
\begin{question}
	Discuss how the ``beating" effect in a coupled oscillator system
	is different than the beating of acoustic waves.
\end{question}

In the case of acoustic waves interacting to create a beat, we're just
talking about two waves summing constructively and destructively. In the
case of CEO, the coupled oscillator system is not described by the addition
of two constant waves, but rather by a second order differential equation.
Rather than two waves being created by entirely different sources, in CEO
the capacitor and inductors are actually causing time delays in the flow
of current as it flows around the circuit and is slowly dissipated by the
resistor. If the system were not operating in a normal mode, these ``beats"
would not be apparent.
















\end{document}
