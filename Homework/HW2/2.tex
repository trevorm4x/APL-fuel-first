\documentclass{article}
\usepackage[margin=0.5in]{geometry}
\usepackage{titlesec}
\usepackage{ifthen}
\usepackage{enumitem}
\usepackage{fancyhdr}
\usepackage{xcolor}

% -------- %
% SECTIONS %
% -------- %
\newcounter{problemnumber}\setcounter{problemnumber}{1}
\titlespacing\section{0pt}{10pt}{0pt}   % Spacing between Problems
\titlespacing\subsection{0pt}{5pt}{0pt} % Spacing between Parts
\newcommand{\problem}[1][-1]{
    \setcounter{partnumber}{1}
    \ifnum#1>0
        \setcounter{problemnumber}{#1}
    \fi
    \section*{Problem \arabic{problemnumber}}
    \stepcounter{problemnumber}
}

\newcounter{partnumber}\setcounter{partnumber}{1}
\newcommand{\ppart}[1][-1]{
    \ifnum#1>0
        \setcounter{partnumber}{#1}
    \fi
    \subsection*{Part \parttype{partnumber}}
    \stepcounter{partnumber}
}

\newenvironment{question}{
    \color{gray}\itshape
    \vspace{5pt}
    \begin{tabular}{|p{0.97\linewidth}}
}{
    \end{tabular}\\[5pt]
}



% ------------- %
% HEADER/FOOTER %
% ------------- %
\setlength\parindent{0pt}
\setlength\headheight{30pt}
\headsep=0.25in
\pagestyle{fancy}
\lhead{\ifthenelse{\thepage=1}
    {\textbf{Trevor Smith} \\ \textbf{\writeday}}
}
\chead{\ifthenelse{\thepage=1}
    {\textbf{\huge{HOMEWORK \hwnumber}}}
    {\textbf{\large{HOMEWORK \hwnumber}}}
}
\rhead{\ifthenelse{\thepage=1}
    {\textbf{{\course}} \\ \textbf{Professor {\prof}}}
}
\cfoot{\thepage}
\renewcommand\headrulewidth{0.4pt}
\renewcommand\footrulewidth{0.4pt}



% ---------- %
% PARAMETERS %
% ---------- %
% \PARTTYPE:
% \Alph   := "Part A, Part B,  ..."
% \alph   := "Part a, Part b,  ..."
% \arabic := "Part 1, Part 2,  ..."
% \Roman  := "Part I, Part II, ..."
\newcommand\parttype{\Roman}

% \COURSE:
\newcommand\course{PHYS 3600}

% \HWNUMBER
\newcommand\hwnumber{2}

% \SEMESTER
\newcommand\semester{Summer 1 2021}

% \PROF
\newcommand\prof{Altunkaynak}

% \WRITEDAY
% \today is date of compilation, replace if writing due date rather than write date
\newcommand\writeday{\today}



%  ------- %
% DOCUMENT %
% -------- %
\begin{document}
\problem
\begin{question}
	Briefly explain why the 4-wire resistance setup eliminates the errors due to contact resistances, thus accurately measuring the resistance of the silicon wafer.
\end{question}

With an ohmmeter, we are not only measuring the resistance of the silicon
wafer but also the resistance of the wires connecting the silicon wafer to 
the ohm meter on either side. However, using the four-wire method we are
accurately measuring the current (because in a closed loop in series with
a voltmeter which draws very little current), and also accurately measuring 
the voltage (because at such low current the voltage travelling through a 
voltmeter drops very little across resistors), resulting in an ideal 
measurement of the wafer itself with no significant impact in I nor V of any
resistances in the system other than the wafer. 

\problem
\begin{question}
	When measuring the resistivity of a semiconductor, what are the crucial dimensions that must be measured when using the following techniques,
	\begin{enumerate}[label=\alph*.]
		\item using the rectangular geometry from the HALL Lab instructions;
		\item using the geometry of the van der Pauw technique.
	\end{enumerate}
 Discuss the main difference in the techniques.
 \end{question}
\begin{enumerate}[label=\alph*.]
	\item
		In the lab we measured all three dimensions, width, length, 
		and thickness, as well as the distance between two nodes on the
		wafer. This was to find the resistivity using cross-sectional
		area and length.
	\item
		The van der Pauw technique only requires the sheet thickness,
		to determine the average resistivity, as it assumes the 
		semiconductor is a sheet.

\end{enumerate}
The first method relies on a simple geometric approach, where the resistance 
between two points on the wafer is taken as the resistivity at each small dA of
area, divided by the cross-sectional area, multiplied by the length. The
second approach assumes the wafer is a sheet, where the width and length are 
much larger than the height, and accounts for any geometry of the wafer 
by mapping it onto an infinite half-plane. The equation is then much more 
complicated, in that latter case.

\problem
\begin{question}
	Write down an equation for the current density (J) in terms of 
	current (I) and cross sectional area (A), including the units.
\end{question}
 $$ J\ (A/m^2) = \frac{I\ (A)}{A\ (m^2)} $$

\problem
\begin{question}
Consider a Hall bar in the shape of a rectangular semiconductor of width (w) and thickness (d) that is placed under the influence of a magnetic field (B) and electrical current (I). Derive an equation for the electric field that compensates for the charge carriers that build up on one side of the Hall bar. 
\end{question}
With an electric field going into the top or bottom of the semiconductor and electrons travelling along the long side of the semiconductor, a $V_H$ will be produced via $V_H = \frac{BI}{ned}$ as given in the lab. In general, $E = \frac{V}{d}$ or in this case $E = \frac{V_H}{w}$ giving finally
$$ E = \frac{BI}{nedw} $$

\newpage
\problem
\begin{question}
Consider a Hall bar experiment using a germanium wafer of thickness d = 1 mm, having a current of I=1.00
mA and a magnetic field of B = 1.00 tesla, which gives a Hall voltage of VH = 1.00 V. \\
a. What is the value of the carrier density? \\
b. Using a value for the mobility of electrons in Ge, what is the value of the resistivity in $\Omega$cm units?
\end{question}

\begin{enumerate}[label=\alph*.]
	\item 
	$$ V_H = \frac{BI}{ned} $$
	Where: \\
	d = 1 mm\\
	I = 1 mA\\
	B = 1 T\\
	$V_H$ = 1 V\\
	e = 1.60217662e-19 coulombs
	$$ n = \frac{BI}{V_Hed} = 6.24e18 m^{-3} $$

	\item
	Mobility of electrons in Ge $\mu = 3900\ cm^2 V^{-1}s^{-1}$ \\

	$$ \rho = \frac{1}{ne\mu} = 2.565e2\ \Omega\ cm $$
\end{enumerate}

\problem
\begin{question}
How does a gaussmeter work?
\end{question}
When a conductor is placed in a magnetic field, electrons are pushed to one side
with a strength proportional to the strength of the magnetic field. By measuring
the voltage on either side of the conductor, called a probe, the gaussmeter can 
measure the strength of the magnetic field.

\problem
\begin{question}
Explain the term Fermi energy (EF)? Where is the position of the Fermi level in 
n-type and p-type semiconductors, relative to the conduction and valence band 
edges?
\end{question}

The Fermi energy is the maximum amount of energy an electron in a given atom can
have at 0 K. For p-type semiconductors, the $E_F$ is close to the valence band, and for n-type semiconductors it is close to the conduction band.

\problem
\begin{question}
	How is a uniform magnetic field created in an electromagnet?
\end{question}
Magnetic fields are created by supplying a current through a conductor.
Further, if we stack the conductor then the effect will also be compounded.
Finally, if we want to produce a uniform magnetic field then ideally we want a
long cylinder made out of looping wire, as the magnetic field inside of the
coil will be uniform. However, if we want to be able to supply a uniform 
magnetic field to something that isn't stuck inside a cylinder, we can put two large coils on either side of a test area, essentially a slice of a cylinder. This will make the magnetic field pretty uniform, especially in a small area between the coils. This is likely how the electromagnet in the lab works.

\problem
\begin{question}
Can you replace the electromagnet with permanent magnets in this particular HALL experiment? Explain your answer.
\end{question}
Permanent magnets have poles, where the field lines are much closer, and would
therefore be very difficult to make a uniform magnetic field out of. There is
also the issue that we need to take a derivative with respect to a changing
magnetic field in order to complete the lab, which would also be difficult to
achieve with permanent magnets.\\

The best way I can think of how to hack it together, would be to tightly tape
a few rectangular permanent magnets together along their longest side, such that
we're kind of imitating the shape of the electromagnet. Then put it close to 
another one of these hacked together fixtures of permanent magnets and hope
some of the field lines go where we want them. We could perhaps move the clumps
further apart to get a smaller and smaller magnetic field, but these would 
probably be pretty weak and pretty non-uniform.









































































































\end{document}
