\documentclass{article}
\usepackage[margin=0.5in]{geometry}
\usepackage{titlesec}
\usepackage{ifthen}
\usepackage{fancyhdr}
\usepackage{xcolor}

% -------- %
% SECTIONS %
% -------- %
\newcounter{problemnumber}\setcounter{problemnumber}{1}
\titlespacing\section{0pt}{10pt}{0pt}   % Spacing between Problems
\titlespacing\subsection{0pt}{5pt}{0pt} % Spacing between Parts
\newcommand{\problem}[1][-1]{
    \setcounter{partnumber}{1}
    \ifnum#1>0
        \setcounter{problemnumber}{#1}
    \fi
    \section*{Problem \arabic{problemnumber}}
    \stepcounter{problemnumber}
}

\newcounter{partnumber}\setcounter{partnumber}{1}
\newcommand{\ppart}[1][-1]{
    \ifnum#1>0
        \setcounter{partnumber}{#1}
    \fi
    \subsection*{Part \parttype{partnumber}}
    \stepcounter{partnumber}
}

\newenvironment{question}{
    \color{gray}\itshape
    \vspace{5pt}
    \begin{tabular}{|p{0.97\linewidth}}
}{
    \end{tabular}\\[5pt]
}



% ------------- %
% HEADER/FOOTER %
% ------------- %
\setlength\parindent{0pt}
\setlength\headheight{30pt}
\headsep=0.25in
\pagestyle{fancy}
\lhead{\ifthenelse{\thepage=1}
    {\textbf{Trevor Smith} \\ \textbf{\writeday}}
}
\chead{\ifthenelse{\thepage=1}
    {\textbf{\huge{HOMEWORK \hwnumber}}}
    {\textbf{\large{HOMEWORK \hwnumber}}}
}
\rhead{\ifthenelse{\thepage=1}
    {\textbf{{\course}} \\ \textbf{Professor {\prof}}}
}
\cfoot{\thepage}
\renewcommand\headrulewidth{0.4pt}
\renewcommand\footrulewidth{0.4pt}



% ---------- %
% PARAMETERS %
% ---------- %
% \PARTTYPE:
% \Alph   := "Part A, Part B,  ..."
% \alph   := "Part a, Part b,  ..."
% \arabic := "Part 1, Part 2,  ..."
% \Roman  := "Part I, Part II, ..."
\newcommand\parttype{\Roman}

% \COURSE:
\newcommand\course{PHYS 3600}

% \HWNUMBER
\newcommand\hwnumber{4}

% \SEMESTER
\newcommand\semester{Summer 1 2021}

% \PROF
\newcommand\prof{Altunkaynak}

% \WRITEDAY
% \today is date of compilation, replace if writing due date rather than write date
\newcommand\writeday{\today}



%  ------- %
% DOCUMENT %
% -------- %
\begin{document}

\problem
\begin{question}
	Light emitted from a conventional source is said to be \emph{\textbf{incoherent}}. Why???
\end{question}

When electrons de-excite they emit photons in random directions at random phases. Thus there is no coherence in 
the direction and phase of the light.

\problem
\begin{question}
	Light traveling in a vacuum will experience changes as it enters a different medium. What properties change 
	and what do not change? Briefly explain.
\end{question}

The speed and wavelength change, but the frequency remains the same, and by consequence the color is the same. 
Intensity will also change.

\problem
\begin{question}
	Can you perform this Speed Out of Luck lab with a 532 nm green laser, instead of the red laser? Do not answer
	yes or no.
\end{question}

Without saying a firm yes or no, technically the lab would work with a different laser, but absorption of light
depends on wavelength for most materials, so different lasers may be less optimal with certain mediums. For water,
an ultraviolet laser or infrared laser would be less ideal. With a green laser would arguably be 
better due to lower absorption.

\problem
\begin{question}
	SOL in a sapphire?
\end{question}
Index of refraction $n=1.77$. This corresponds to the ratio of the speed of light in vacuum to the speed of
light in that material. $2.998e8*/1.77=1.693e8$ m/s

\problem
\begin{question}
	List and briefly describe the important parts of a fiber optic cable.
\end{question}

\begin{itemize}
	\item Jacket/Sheathing just an outside cover
	\item Kevlar protects the cable and prevents damage while the cable is pulled
	\item Cladding a reflective cladding that goes around the core and keeps light from stopping
	\item Core a continuous strand of super thin glass that is roughly the same size as a human hair. 
		Serves as the medium through which the light moves.
	\item Boot begins the transition from cable to connector, bends more comfortably to protect the connection.
	\item Connector plugs in to the equipment.
	\item Ferrule a protuding portion of a fiber connector, houses the end of the fiber to align it with 
		another cable.
\end{itemize}

\problem
\begin{question}
	Describe \textbf{\emph{total internal reflection}}.
\end{question}

This means that light does not leave the medium through some interface, and that it will be completely reflected 
back inwards. More specifically, this happens which going from a more dense to a less dense medium when the angle
of incidence is greater than the critical angle.

\problem
\begin{question}
	What is the value of the critical angle for total internal reflection in a diamond crystal? 
\end{question}
The critical angle is the incident angle at which the refracted angle is 90.
Assuming the medium the refracted ray is traversing is vacuum, or $n_2=1$, and $n_1=2.3778$.
$$ \theta_{crit} = arcsin\left(\frac{n_2}{n_1}\right) =  arcsin(1/2.3778) = 24.8\ degrees $$

\problem
\begin{question}
	Suppose you are measuring the speed of light in air by splitting a laser beam into two beams that are 
	traveling different distances and you measure the time delay between the two beams. If you did the 
	same experiment in a vacuum, what would be the percentage difference in the computed speed of light? 
\end{question}
This question is super unclear, but I think it means what's the difference in speed of light in vacuum compared 
to speed of light in air??? In which case, it's going to be equal to n-1 (since in a vaccuum n is 1), for air that
is going to give us $0.0293\%$.

\problem
\begin{question}
	How much longer than in air would it take for light to pass through a 1.0000 cm diamond?
\end{question}
$$ t = \frac{d}{v} $$
$$ v = \frac{c}{n} $$
$$ t = \frac{dn}{c} $$
$$ \Delta t = \frac{d(n_{diamond} - n_{air}}{c} $$
$$ \Delta t = \frac{1.0000e-2\ m(2.417- 1.000}{2.9979\ m/s} $$
$$ \Delta t = 0.0047266\ seconds $$



\end{document}
